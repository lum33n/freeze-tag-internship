\subsection{Contexte Sociale et Institutionnel}

\subsubsection{Environnement social}

Durant mon stage, j'ai travaillé dans une salle avec 6 autres stagiaires avec qui nous avons pu partager nos sujets, nos expériences, nos succès et cela fut très intéressant. Tous les midis, nous avons mangé avec les doctorants dans une ambiance très conviviale. 
Dans le cadre des recherches, j'aurais principalement parlé avec Cyril Gavoille et Clément Legrand-Duchesne respectivement mon encadrant et mon co-encadrant. Chaque chercheur que j'ai pu rencontrer travaillait sur plusieurs sujets à la fois mais tous sont ouverts aux sujets des autres créant un travail très collectif.

\subsubsection{organisation} Le LaBRI est constitué de plusieurs équipes au seins de différents départements. J'étais personnellement au sein des équipes "Algorithme distribué" et combinatoire dans le département "CombAlgo". Chaque équipe a un séminaire chaque semaine auquel nous pouvons assister pour découvrir un nouveau sujet, une nouvelle méthode de réflexion. De plus, en tant que stagiaire, tout au long on nous aura présenté des séminaires spéciaux pour les stagiaires où l'on nous présentait des nouvelles notions activement recherchées au LaBRI. Il était également possible d'assister à des soutenances de thèse.